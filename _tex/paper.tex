% Options for packages loaded elsewhere
\PassOptionsToPackage{unicode}{hyperref}
\PassOptionsToPackage{hyphens}{url}
\PassOptionsToPackage{dvipsnames,svgnames,x11names}{xcolor}
%
\documentclass[
]{agujournal2019}

\usepackage{amsmath,amssymb}
\usepackage{iftex}
\ifPDFTeX
  \usepackage[T1]{fontenc}
  \usepackage[utf8]{inputenc}
  \usepackage{textcomp} % provide euro and other symbols
\else % if luatex or xetex
  \usepackage{unicode-math}
  \defaultfontfeatures{Scale=MatchLowercase}
  \defaultfontfeatures[\rmfamily]{Ligatures=TeX,Scale=1}
\fi
\usepackage{lmodern}
\ifPDFTeX\else  
    % xetex/luatex font selection
\fi
% Use upquote if available, for straight quotes in verbatim environments
\IfFileExists{upquote.sty}{\usepackage{upquote}}{}
\IfFileExists{microtype.sty}{% use microtype if available
  \usepackage[]{microtype}
  \UseMicrotypeSet[protrusion]{basicmath} % disable protrusion for tt fonts
}{}
\makeatletter
\@ifundefined{KOMAClassName}{% if non-KOMA class
  \IfFileExists{parskip.sty}{%
    \usepackage{parskip}
  }{% else
    \setlength{\parindent}{0pt}
    \setlength{\parskip}{6pt plus 2pt minus 1pt}}
}{% if KOMA class
  \KOMAoptions{parskip=half}}
\makeatother
\usepackage{xcolor}
\setlength{\emergencystretch}{3em} % prevent overfull lines
\setcounter{secnumdepth}{5}
% Make \paragraph and \subparagraph free-standing
\makeatletter
\ifx\paragraph\undefined\else
  \let\oldparagraph\paragraph
  \renewcommand{\paragraph}{
    \@ifstar
      \xxxParagraphStar
      \xxxParagraphNoStar
  }
  \newcommand{\xxxParagraphStar}[1]{\oldparagraph*{#1}\mbox{}}
  \newcommand{\xxxParagraphNoStar}[1]{\oldparagraph{#1}\mbox{}}
\fi
\ifx\subparagraph\undefined\else
  \let\oldsubparagraph\subparagraph
  \renewcommand{\subparagraph}{
    \@ifstar
      \xxxSubParagraphStar
      \xxxSubParagraphNoStar
  }
  \newcommand{\xxxSubParagraphStar}[1]{\oldsubparagraph*{#1}\mbox{}}
  \newcommand{\xxxSubParagraphNoStar}[1]{\oldsubparagraph{#1}\mbox{}}
\fi
\makeatother


\providecommand{\tightlist}{%
  \setlength{\itemsep}{0pt}\setlength{\parskip}{0pt}}\usepackage{longtable,booktabs,array}
\usepackage{calc} % for calculating minipage widths
% Correct order of tables after \paragraph or \subparagraph
\usepackage{etoolbox}
\makeatletter
\patchcmd\longtable{\par}{\if@noskipsec\mbox{}\fi\par}{}{}
\makeatother
% Allow footnotes in longtable head/foot
\IfFileExists{footnotehyper.sty}{\usepackage{footnotehyper}}{\usepackage{footnote}}
\makesavenoteenv{longtable}
\usepackage{graphicx}
\makeatletter
\def\maxwidth{\ifdim\Gin@nat@width>\linewidth\linewidth\else\Gin@nat@width\fi}
\def\maxheight{\ifdim\Gin@nat@height>\textheight\textheight\else\Gin@nat@height\fi}
\makeatother
% Scale images if necessary, so that they will not overflow the page
% margins by default, and it is still possible to overwrite the defaults
% using explicit options in \includegraphics[width, height, ...]{}
\setkeys{Gin}{width=\maxwidth,height=\maxheight,keepaspectratio}
% Set default figure placement to htbp
\makeatletter
\def\fps@figure{htbp}
\makeatother
% definitions for citeproc citations
\NewDocumentCommand\citeproctext{}{}
\NewDocumentCommand\citeproc{mm}{%
  \begingroup\def\citeproctext{#2}\cite{#1}\endgroup}
\makeatletter
 % allow citations to break across lines
 \let\@cite@ofmt\@firstofone
 % avoid brackets around text for \cite:
 \def\@biblabel#1{}
 \def\@cite#1#2{{#1\if@tempswa , #2\fi}}
\makeatother
\newlength{\cslhangindent}
\setlength{\cslhangindent}{1.5em}
\newlength{\csllabelwidth}
\setlength{\csllabelwidth}{3em}
\newenvironment{CSLReferences}[2] % #1 hanging-indent, #2 entry-spacing
 {\begin{list}{}{%
  \setlength{\itemindent}{0pt}
  \setlength{\leftmargin}{0pt}
  \setlength{\parsep}{0pt}
  % turn on hanging indent if param 1 is 1
  \ifodd #1
   \setlength{\leftmargin}{\cslhangindent}
   \setlength{\itemindent}{-1\cslhangindent}
  \fi
  % set entry spacing
  \setlength{\itemsep}{#2\baselineskip}}}
 {\end{list}}
\usepackage{calc}
\newcommand{\CSLBlock}[1]{\hfill\break\parbox[t]{\linewidth}{\strut\ignorespaces#1\strut}}
\newcommand{\CSLLeftMargin}[1]{\parbox[t]{\csllabelwidth}{\strut#1\strut}}
\newcommand{\CSLRightInline}[1]{\parbox[t]{\linewidth - \csllabelwidth}{\strut#1\strut}}
\newcommand{\CSLIndent}[1]{\hspace{\cslhangindent}#1}

\usepackage{url} %this package should fix any errors with URLs in refs.
\usepackage{lineno}
\usepackage[inline]{trackchanges} %for better track changes. finalnew option will compile document with changes incorporated.
\usepackage{soul}
\linenumbers
\makeatletter
\@ifpackageloaded{caption}{}{\usepackage{caption}}
\AtBeginDocument{%
\ifdefined\contentsname
  \renewcommand*\contentsname{Table of contents}
\else
  \newcommand\contentsname{Table of contents}
\fi
\ifdefined\listfigurename
  \renewcommand*\listfigurename{List of Figures}
\else
  \newcommand\listfigurename{List of Figures}
\fi
\ifdefined\listtablename
  \renewcommand*\listtablename{List of Tables}
\else
  \newcommand\listtablename{List of Tables}
\fi
\ifdefined\figurename
  \renewcommand*\figurename{Figure}
\else
  \newcommand\figurename{Figure}
\fi
\ifdefined\tablename
  \renewcommand*\tablename{Table}
\else
  \newcommand\tablename{Table}
\fi
}
\@ifpackageloaded{float}{}{\usepackage{float}}
\floatstyle{ruled}
\@ifundefined{c@chapter}{\newfloat{codelisting}{h}{lop}}{\newfloat{codelisting}{h}{lop}[chapter]}
\floatname{codelisting}{Listing}
\newcommand*\listoflistings{\listof{codelisting}{List of Listings}}
\makeatother
\makeatletter
\makeatother
\makeatletter
\@ifpackageloaded{caption}{}{\usepackage{caption}}
\@ifpackageloaded{subcaption}{}{\usepackage{subcaption}}
\makeatother

\ifLuaTeX
  \usepackage{selnolig}  % disable illegal ligatures
\fi
\usepackage{bookmark}

\IfFileExists{xurl.sty}{\usepackage{xurl}}{} % add URL line breaks if available
\urlstyle{same} % disable monospaced font for URLs
\hypersetup{
  pdftitle={Cloud-Optimized HDF5 for NASA's ICESat-2 Mission},
  pdfauthor={Luis A. Lopez; Andrew P. Barrett; Amy Steiker},
  pdfkeywords={cloud-native, cloud, HDF5, NASA, ICESat-2},
  colorlinks=true,
  linkcolor={blue},
  filecolor={Maroon},
  citecolor={Blue},
  urlcolor={Blue},
  pdfcreator={LaTeX via pandoc}}


\journalname{Geophysical Research Letters}

\draftfalse

\begin{document}
\title{Cloud-Optimized HDF5 for NASA's ICESat-2 Mission}

\authors{Luis A. Lopez\affil{1}, Andrew P. Barrett\affil{1}, Amy
Steiker\affil{1}}
\affiliation{1}{CIRES, National Snow and Ice Data Center, University of
Colorado, Boulder., Boulder, CO, USA}



\begin{abstract}
The Hierarchical Data Format (HDF) is a common archival format for
n-dimensional scientific data; it has been utilized to store valuable
information from astrophysics to earth sciences and everything in
between. As flexible and powerful as HDF can be, it comes with big
tradeoffs when it's accessed from remote storage systems, mainly because
the file format and the client I/O libraries were designed for local and
supercomputing workflows. As scientific data and workflows migrate to
the cloud , efficient access to data stored in HDF format is a key
factor that will accelerate or slow down ``science in the cloud'' across
all disciplines. We present an implementation of recently available
features in the HDF5 stack that results in performant access to HDF from
remote cloud storage. This performance is on par with modern
cloud-native formats like Zarr but with the advantage of not having to
reformat data or generate metadata sidecar files (DMR++, Kerchunk). Our
benchmarks also show potential cost-savings for data producers if their
data are processed using cloud-optimized strategies.
\end{abstract}





\section{Section Heading}\label{section-heading}

Lorem ipsum dolor sit amet, consectetur adipiscing elit. Vestibulum
hendrerit facilisis velit sit amet malesuada. Phasellus ornare nibh
augue, maximus sodales ex tristique vitae. Vivamus non sollicitudin
orci, aliquam placerat metus. Maecenas volutpat orci felis, vel finibus
urna consectetur sed. Integer in dui ac dui mollis imperdiet. Quisque
sed dapibus nibh. Aenean non luctus leo. Phasellus luctus mauris id
aliquet dictum. Aliquam fermentum semper massa, vel dignissim nibh
dictum et. See Hubbard et al. (2021).

Phasellus interdum tincidunt ex, a euismod massa pulvinar at. Ut
fringilla ut nisi nec volutpat. Morbi imperdiet congue tincidunt.
Vivamus eget rutrum purus. Etiam et pretium justo. Donec et egestas sem.
Donec molestie ex sit amet viverra egestas. Nullam justo nulla,
fringilla at iaculis in, posuere non mauris. Ut eget imperdiet elit.

In luctus mauris vitae imperdiet luctus. Morbi volutpat ligula ut tortor
fermentum, eu ornare felis luctus. Donec semper diam vitae mattis
posuere. Suspendisse facilisis purus nisi, sit amet egestas ex tempor
ut. Cras tortor nulla, euismod at fermentum vel, dictum vel justo.
Aenean commodo interdum diam nec placerat. Nunc vestibulum felis at est
tincidunt, at euismod dui vestibulum. Nulla venenatis tortor at auctor
iaculis. Donec consectetur neque ut sagittis ornare. Nullam pharetra
felis tempor suscipit efficitur. Curabitur nibh ex, euismod at congue
hendrerit, egestas id mi. Duis porttitor neque in commodo elementum.
Fusce vitae fermentum nisi, euismod viverra augue. Curabitur at mi
pretium, accumsan purus nec, tempus turpis.

Donec non semper dui, quis aliquet est. Quisque quis sapien at massa
ultricies egestas. Duis consequat ultricies erat, a pulvinar nisl
vestibulum id. Sed tristique turpis ligula, et tempor lectus iaculis at.
Vivamus commodo sapien ac turpis vestibulum dapibus. Morbi tristique
arcu metus, et laoreet nisi varius nec. Pellentesque habitant morbi
tristique senectus et netus et malesuada fames ac turpis egestas. Fusce
sit amet nisl at mauris suscipit aliquet. Nulla vitae dignissim urna.
Suspendisse sit amet arcu vitae magna blandit mattis. Vivamus convallis
efficitur pulvinar. Sed cursus elit nulla. Sed porta, arcu a euismod
pretium, odio dui lacinia lacus, ac vulputate nulla augue eget ex.
Nullam consequat ligula sit amet mattis aliquam. Nulla risus urna,
ultrices vel ullamcorper id, ornare viverra nunc.

Nunc in lobortis lacus. Duis maximus urna leo, varius sodales arcu
interdum nec. Pellentesque imperdiet dolor in leo eleifend dapibus. Ut
dapibus, lectus non viverra gravida, ipsum ex faucibus tellus, quis
iaculis risus tellus eget augue. Nullam a viverra est. Cras velit nisi,
interdum in lacus at, vehicula mattis elit. Curabitur eu viverra purus.
Proin pellentesque, metus vitae congue convallis, lorem metus feugiat
mi, sit amet auctor purus ligula bibendum ante. Nam id justo
scelerisque, rhoncus lectus in, fermentum libero. Donec tincidunt
egestas ex ac eleifend. Cras faucibus ipsum a nunc faucibus fermentum.
Integer et maximus lacus. Nam dictum nibh id viverra convallis.

\section{Acknowledgments}\label{acknowledgments}

Phasellus interdum tincidunt ex, a euismod massa pulvinar at. Ut
fringilla ut nisi nec volutpat. Morbi imperdiet congue tincidunt.
Vivamus eget rutrum purus. Etiam et pretium justo. Donec et egestas sem.
Donec molestie ex sit amet viverra egestas. Nullam justo nulla,
fringilla at iaculis in, posuere non mauris. Ut eget imperdiet elit.

\section{Open research}\label{open-research}

Phasellus interdum tincidunt ex, a euismod massa pulvinar at. Ut
fringilla ut nisi nec volutpat. Morbi imperdiet congue tincidunt.
Vivamus eget rutrum purus. Etiam et pretium justo. Donec et egestas sem.
Donec molestie ex sit amet viverra egestas. Nullam justo nulla,
fringilla at iaculis in, posuere non mauris. Ut eget imperdiet elit.

\section*{References}\label{references}
\addcontentsline{toc}{section}{References}

\phantomsection\label{refs}
\begin{CSLReferences}{1}{0}
\vspace{1em}

\bibitem[\citeproctext]{ref-Hubbard2021}
Hubbard, B., Christoffersen, P., Doyle, S. H., Chudley, T. R.,
Schoonman, C. M., Law, R., \& Bougamont, M. (2021). Borehole-based
characterization of deep mixed-mode crevasses at a greenlandic outlet
glacier. \emph{{AGU} Advances}, \emph{2}(2).
\url{https://doi.org/10.1029/2020av000291}

\end{CSLReferences}




\end{document}
